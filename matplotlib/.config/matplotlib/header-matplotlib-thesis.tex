\usepackage{amsmath}
\usepackage{amssymb}
\usepackage{mathtools}
\usepackage{fontspec}
\usepackage{xcolor}
% \setmainfont[BoldFont=Akkurat Office]{Akkurat Light Office}
% \setsansfont[BoldFont=Akkurat Office]{Akkurat Light Office}
% \setmonofont[BoldFont=Fira Mono, Scale=MatchLowercase]{Fira Mono}
\usepackage[
  math-style=ISO,    % ┐
  bold-style=ISO,    % │
  sans-style=italic, % │ ISO-Standard folgen
  nabla=upright,     % │
  partial=upright,   % ┘
  warnings-off={%          % ┐
    mathtools-colon,       % │ unnötige Warnungen ausschalten
    mathtools-overbracket, % │
  },                       % ┘
]{unicode-math}
% \setmathfont{Tex Gyre Pagella Math}
\usepackage[
  locale=DE,                   % deutsche Einstellungen
  separate-uncertainty=true,   % immer Fehler mit \pm
  per-mode=symbol-or-fraction, % / in inline math, fraction in display math
]{siunitx}

\AtBeginDocument{%
  \sisetup{%
    math-rm=\mathrm,
  }
}
